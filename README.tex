\documentclass[a4paper,12pt]{article}
\usepackage[margin=1in]{geometry}
\usepackage[utf8]{inputenc}
\usepackage[english,vietnamese]{babel}
\usepackage{textcomp}
\usepackage{amssymb}
\usepackage{graphicx}
\usepackage{hyperref}
\def\UrlBreaks{\do\/\do-}
\title{BẢN THUYẾT MINH PHẦN MỀM SÁNG TẠO}
\author{}
\date{}

\begin{document}
\begin{center}
HỘI THI TIN HỌC TRẺ THÀNH PHỐ HÀ NỘI LẦN THỨ XXXIII, NĂM 2018
\rule{0.8\textwidth}{0.4pt}
\end{center}
{\let\newpage\relax\maketitle}
\section*{A. THÔNG TIN CHUNG}
\subsection*{1. Giới thiệu sản phẩm}
Tên sản phẩm: Mê cung Khốc liệt (Brutal Maze)\\
Công cụ sử dụng (ngôn ngữ lập trình): Python\\
Cấu hình cài đặt: Máy tính điện tử với Windows, macOS, GNU/Linux, *BSD hoặc một
số hệ điều hành giống-Unix khác\\
Đăng kí bảng thi:\hspace{1cm}D1 $\square$\hspace{1in}D2 $\square$
\hspace{1in}D3 $\boxtimes$\\
Thuộc đoàn (tên quận/huyện/thị xã): Ba Đình

\subsection*{2. Tên tác giả, nhóm tác giả}
Họ và tên thí sinh: Nguyễn Gia Phong\\
Ngày tháng năm sinh: 26/05/2000\\
Học sinh lớp 12A1\hspace{1in}trường THPT Phạm Hồng Thái\\
Số điện thoại liên lạc: (024) 38350837\hspace{1in}Email: vn.mcsinyx@gmail.com

\section*{B. MÔ TẢ SẢN PHẨM}
\subsection*{1. Yêu cầu cơ sở hạ tầng để triển khai ứng dụng}
\begin{itemize}
  \item Phần cứng: Máy tính điện tử với kết nối Internet
  \item Phần mềm:
    \begin{itemize}
      \item Hệ điều hành: Microsoft Windows, Apple macOS, GNU/Linux, các *BSD
        hoặc một số Unix-like khác
      \item Thư viện: Python, pip\footnote{\url{https://pip.pypa.io/}},
        pygame\footnote{\url{https://www.pygame.org/wiki/about}} và
        appdirs\footnote{\url{https://github.com/ActiveState/appdirs}}
    \end{itemize}
\end{itemize}

\subsection*{2. Sản phẩm được phát triển ước tính trong khoảng thời gian}
Số tháng: 6 tháng (từ 10/2017 đến 03/2018)

\subsection*{3. Phạm vi ứng dụng}
Mê cung Khốc liệt hướng đến hai đối tượng sử dụng chính:

\begin{description}
  \item[Người chơi] Là trò chơi hành động đơn giản với giao diện trực quan, Mê
    cung Khốc liệt phù hợp cho mọi lứa tuổi.
  \item[Người yêu thích lập trình] Hỗ trợ TCP/IP socket cho điều từ chương
    trình bên ngoài, trò chơi có thể được sử dụng trong học lập trình bot.
\end{description}

\subsection*{4. Hướng dẫn cài đặt và sử dụng}
Mê cung Khốc liệt có thể chạy trên các phiên bản hiện hành của cả Python 2 và
3. Trước khi cài đặt trò chơi, người dùng cần cài Python, pip và các thư viện
pygame theo các bước sau:
\begin{enumerate}
  \item Cài đặt Python, setuptools và pip:
    \begin{itemize}
      \item Cho Windows: Chạy file \verb|dist/python-*.exe| được đóng gói kèm
        sản phẩm.
      \item Cho macOS: Chạy file \verb|dist/python-*-macosx*.pkg| được đóng gói
        kèm sản phẩm.
      \item Cho các Unix-like khác: Sử dụng phần mềm quản lí gói của bản phân
        phối để cài các gói \verb|python|, \verb|python-pip|,
        \verb|python-setuptools| (tên chính xác các gói phụ thuộc vào bản phân
        phối)~\cite{nixpip}.
    \end{itemize}
  \item Cài đặt các thư viện (nếu có đường truyền Internet có thể bỏ qua bước
    này và chuyển sang bước 3):
    \begin{itemize}
      \item appdirs: Trong Command Prompt/PowerShell (với Windows) hoặc
        \mbox{Terminal/Console} (trong các hệ điều hành họ Unix), chạy lệnh
        \verb|pip install dist/appdirs-*.whl| (với các tệp \verb|dist/*.whl|
        được đóng gói trong thư mục sản phẩm).
      \item pygame: Chạy lệnh \verb|pip install dist/pygame-*-<nền tảng>.whl|,
        với \emph{nền tảng} là hệ điều hành và CPU.
    \end{itemize}
  \item Cài đặt trò chơi: Sử dụng lệnh \verb|pip install dist/brutalmaze-*.whl|.
    Với máy tính có kết nối Internet, \verb|pip install brutalmaze| sẽ tự động
    cài đặt cả các thư viện phụ thuộc ở bước 2.
\end{enumerate}

Sau khi cài đặt, trò chơi có thể được mở trong launcher của hệ điều hành với
tên \verb|brutalmaze|. Mặc định, các phím mũi tên (\textuparrow \textleftarrow
\textdownarrow \textrightarrow) được sử dụng để di chuyển, chuột trái để bắn và
chuột phải để chém và đỡ đạn.

\subsection*{5. Mô tả các tính năng cơ bản của sản phẩm}
Mê cung Khốc liệt là trò chơi hành động tiết tấu nhanh. Người chơi phải điều
khiển hình tam giác bất lực chiến đấu với những hình vuông hòng thoát khỏi mê
cung. Các tính năng nổi bật của trò chơi:

\newpage
\begin{itemize}
  \item[-] Mê cung vô tận được tạo một cách ngẫu nhiên.
  \item[-] Cửa sổ trò chơi có thể thu/phóng ở hầu hết mọi tỷ lệ\footnote{Thực tế
    tỉ lệ nằm ngoài khoảng 1:16 đến 16:1 sẽ gây khó khăn cho việc tạo mê cung.}.
  \item[-] Bot (các hình vuông) phong phú với nhiều khả năng như vô hình, làm
    bất động, nhân bản, \ldots
  \item[-] Engine vật lí tương đối thực với quán tính của chuyển động và mức sát
    thương dựa trên khoảng cách.
  \item[-] Dễ dàng tuỳ chỉnh qua lựa chọn dòng lệnh hoặc tệp INI.
  \item[-] Hỗ trợ điều khiển qua TCP/IP socket.
\end{itemize}
Các tuỳ chọn đơn giản có thể được xác định qua giao diện dòng lệnh:
\begin{description}
  \item[-h, -{}-help] in trợ giúp và thoát
  \item[-v, -{}-version] in phiên bản và thoát
  \item[-{}-write-config \lbrack PATH\rbrack ] in tuỳ chỉnh mặc định vào tệp PATH
    và thoát, nếu PATH không được chỉ định, in ra stdout
  \item[-c, -{}-config PATH] nạp tuỳ chỉnh từ tệp PATH
  \item[-s, -{}-size] chỉ định kích thước XxY cho trò chơi
  \item[-f, -{}-max-fps FPS] chỉ định số khung hình tối đa trong một giây
  \item[-{}-\lbrack un\rbrack mute] tắt âm thanh
  \item[-{}-music-volume VOL] đặt âm lượng nhạc nền (trong khoảng 0.0 đến 1.0)
  \item[-{}-\lbrack no-\rbrack server] bật/tắt server socket
  \item[-{}-host] đặt địa chỉ server
  \item[-{}-port] đặt cổng server
  \item[-t, -{}-timeout TIMEOUT] đặt thời gian chờ trước khi server ngắt kết nối
    khỏi client khi không nhận được hồi đáp (tính theo giây)
  \item[-{}-head\lbrack less\rbrack] bật/tắt cửa sổ trờ chơi khi chạy trong chế
    độ server
\end{description}

Tệp tuỳ chỉnh hỗ trợ thêm các lựa chọn về điều khiển. Tệp mặc định có dạng như
sau:
\begin{verbatim}
[Graphics]
# Độ rộng và độ dài khung hình (tính theo pixel)
Screen width: 640
Screen height: 480
# FPS tối đa không nên vượt quá tốc độ nháy của màn hình
# để tránh lãng phí tài nguyên
Maximum FPS: 60
\end{verbatim}

\newpage
\begin{verbatim}
[Sound]
# Tắt âm thanh (yes cho có, no cho không)
Muted: no
# Âm lượng nhạc nền, trong khoảng 0.0 đến 1.1
Music volume: 1.0

# Các phím điều khiển hợp lệ được liệt kê trong
# http://www.pygame.org/docs/ref/key.html
# (không tính phần K_). Tổ hợp phím không được hỗ trợ.
# Chuột trái, giữa, phải lần lượt là Mouse1 đến Mouse3.
[Control]
# Trò chơi mới
New game: F2
# Tạm dừng trò chơi
Toggle pause: p
# Tắt/bật tiếng
Toggle mute: m
# Di chuyển sang trái, phải, trên. dưới
Move left: Left
Move right: Right
Move up: Up
Move down: Down
# Bắn và chém (đồng thời đỡ đạn)
Long-range attack: Mouse1
Close-range attack: Mouse3

[Server]
# Bật server sẽ vô hiệu hoá điều khiển trực tiếp
Enable: no
# Địa chỉ server socket
Host: localhost
Port: 8089
# Thời gian chờ trước khi server ngắt kết nối khỏi client
# khi không nhận được hồi đáp (tính theo giây)
Timeout: 1.0
# Bật/tắt cửa sổ trờ chơi khi chạy trong chế độ server
Headless: no
\end{verbatim}
Các tuỳ chỉnh được nạp theo thứ tự sau:
\begin{enumerate}
  \item[0.] Tuỳ chọn mặc định
  \item Tệp tuỳ chỉnh hệ thống và của người dùng\footnote{Lần lượt được liệt kê
    trong \texttt{brutalmaze -h} ở thẻ \texttt{-{}-config}.}
  \item Tệp INI được chỉ định bởi thẻ \verb|--config|
  \item Các tuỳ chọn trong dòng lệnh
\end{enumerate}
Các giá trị nạp sau sẽ đè lên giá trị trước.

\newpage
Khi được khởi động với thẻ \verb|--server| hoặc chỉ định trong tệp tuỳ chỉnh
(\verb|Enable: yes| trong mục \verb|[Server]|), Mê cung Khốc liệt sẽ mở một
TCP/IPv4 socket tại địa chỉ đã chọn. Trước khi gửi dữ liệu trò chơi, server sẽ
gửi độ dài $l$ của dữ liệu dài 7 byte (ví dụ \texttt{0000531}). Dữ liệu có dạng
chung như sau:
\begin{verbatim}
<Số dòng mê cung (nh)> <Số kẻ thù (ne)> <Số đạn (nb)> <Điểm>
<nh dòng mô tả phần mê cung được hiển thị>
<Một dòng mô tả tam giác>
<ne dòng mô tả các kẻ thù (hình vuông)>
<nb dùng mô tả các viên đạn đang bay>
\end{verbatim}
\begin{description}
  \item[Mê cung] Biểu diễn mê cung là bản đồ bit (0 và 1) dài nh dòng và rộng nw
    cột. Mỗi ô tương ứng với 100 đơn vị độ dài.
  \item[Tam giác] Biểu diễn bởi 6 giá trị:
    \begin{description}
      \item[Màu sắc] Một kí tự biểu thị thể lực nhân vật
      \item[Hoành độ] Số nguyên trong đoạn $[0, nw\times 100]$
      \item[Tung độ] Số nguyên trong đoạn $[0, nh\times 100]$
      \item[Góc] Hướng ngắm bắn của nhân vật (theo độ)
      \item[Có thể tấn công] \texttt{1} nếu có và \texttt{0} nếu không
      \item[Có thể hồi sức] \texttt{1} nếu có và \texttt{0} nếu không
    \end{description}
  \item[Hình vuông] biểu diễn bởi 4 giá trị:
    \begin{description}
      \item[Màu sắc] Một kí tự biểu thị thể lực nhân vật
      \item[Hoành độ] Số nguyên trong đoạn $[0, nw\times 100]$
      \item[Tung độ] Số nguyên trong đoạn $[0, nh\times 100]$
      \item[Góc] Hướng ngắm bắn của nhân vật (theo độ)
    \end{description}
  \item[Đạn] biểu diễn bởi 4 giá trị:
    \begin{description}
      \item[Màu sắc] Một kí tự biểu thị sức sát thương (trong [0, 1])
      \item[Hoành độ] Số nguyên trong đoạn $[0, nw\times 100]$
      \item[Tung độ] Số nguyên trong đoạn $[0, nh\times 100]$
      \item[Góc] Hướng bay (theo độ)
    \end{description}
\end{description}

\newpage
Cần chú ý rằng trong trò chơi, trục hoành hướng xuống dưới kéo theo sự đảo ngược
chiều dương của góc lượng giác:
\begin{center}
\includegraphics[width=.5\linewidth]{wiki/pics/unitCircleDegrees.png}
\end{center}

Thể lực các nhân vật theo màu được biểu diễn bằng các kí tự như sau:
\begin{center}
\begin{tabular*}{0.6\linewidth}{@{\extracolsep{\fill}} l|l|l|l|l|l|l}
  Thể lực & 5 & 4 & 3 & 2 & 1 & 0 \\\hline
  Vàng & & & a & b & c & 0 \\
  Cam & & & d & e & f & 0 \\
  Nâu & & & g & h & i & 0 \\
  Lục & & & j & k & l & 0 \\
  Lam & & & m & n & o & 0 \\
  Tím & & & p & q & r & 0 \\
  Đỏ & & & s & t & u & 0 \\
  Xám & v & w & x & y & z & 0
\end{tabular*}
\end{center}

Sức sát thương của đạn cũng được kí hiệu tương tự (ứng với 1, $\frac{2}{3}$,
$\frac{1}{3}$ và 0), riêng đạn xám của tam giác dùng 4 kí tự \texttt{vwx0}.

Tương ứng, server nhận tối đa 7 byte theo dạng \texttt{<Hướng di chuyển> <Góc
ngắm> <Cách tấn công>}:
\begin{description}
  \item[Hướng di chuyển] được mã hoá theo bảng sau:

    \begin{tabular}{l|c c c}
      Hướng & Trái & || & Phải\\\hline
      Trên & 0 & 1 & 2\\
      || & 3 & 4 & 5\\
      Dưới & 6 & 7 & 8
    \end{tabular}

  \item[Góc] tính theo độ, làm tròn về số nguyên trong [0, 360).
  \item[Cách tấn công] \texttt{0} chờ, \texttt{1} bắn, \texttt{2} chém.
\end{description}

\newpage
Như vậy, client sẽ phải thực hiện các bước sau:
\begin{enumerate}
  \item Tạo TCP/IP socket và kết nối với server
  \item Nhận độ dài $l$ của dữ liệu trò chơi, nếu $l < 0$ thì đóng socket và
    thoát
  \item Nhận dữ liệu dài $l$ byte
  \item Xử lí dữ liệu
  \item Gửi hướng dẫn điều khiển tam giác đến server và quay lại bước 2
\end{enumerate}

Client sẽ được đánh giá dựa trên số điểm đạt được và thời gian đạt số điểm đó
(càng ngắn càng tốt). Để thuận tiện, server sẽ in các thông số này ra stdout.

\subsection*{6. Tự đánh giá tiềm năng ứng dụng của sản phẩm}
Trò chơi có thể ứng dụng trong giải trí đơn thuần cũng như học hỏi, luyện tập
lập trình bot trong thời gian thực. Có thể dễ nhận thấy rằng, chiến thuật tốt
đòi hỏi phải xử lí các vật thể trong dòng thời gian để phán đoán cũng như lập ra
bước đi tiếp theo. \emph{Học máy} (machine learning) có thể ứng dụng trong bài
toán này bởi thực tế không hề có cách giải hoàn hảo.

Mặt khác, là phần mềm tự do\footnote{Trò chơi được phát hành theo giấy phép GNU
AGPL bản 3 tại \url{https://github.com/McSinyx/brutalmaze}.} với dung
lượng mã nguồn tương đổi nhỏ (khoảng 1500 dòng kể cả chú thích và dòng trống) và
tài liệu tương đối đầy đủ, Mê cung Khốc liệt còn có thể dùng như ví dụ điển hình
cho lập trình trò chơi đơn giản, xử lí tuỳ chỉnh và socket trong Python.

\subsection*{7. Tự đánh giá những mặt tồn tại chưa được giải quyết}
Trò chơi còn tồn tại một số vấn đề như sau:
\begin{itemize}
  \item Thuật toán tạo mê cung còn bất cập: di chuyển theo hướng Tây Bắc sẽ
    không bao giờ gặp đường cùng.~\cite{algo}
  \item Đồ hoạ còn chưa hấp dẫn.
\end{itemize}

\section*{C. KẾT LUẬN}
\subsection*{1. Hướng phát triển của sản phẩm trong tương lai}
Giải quyết các mặt tồn tại (về về đồ hoạ và thuật toán tạo mê cung); mặt khác có
thể viết thêm phiên bản cho điện thoại thông minh và cho web server.
\subsection*{2. Nguyện vọng trong tương lai}
Trò chơi được sử dụng rộng rãi và có ích cho cộng đồng trong cả phương diện giải
trí cũng như học tập.

\renewcommand{\refname}{D. TÀI LIỆU THAM KHẢO}
\begin{thebibliography}{99}
  \bibitem{nixpip} \url{https://packaging.python.org/guides/installing-using-linux-tools/\#installing-pip-setuptools-wheel-with-linux-package-managers}
  \bibitem{algo} Tạo mê cung bằng cây nhị phân:
  \url{http://weblog.jamisbuck.org/2011/2/1/maze-generation-binary-tree-algorithm}
\end{thebibliography}

\raggedleft{Ba Đình, ngày 04 tháng 04 năm 2018}

\raggedleft{Chữ kí của tác giả/nhóm tác giả}
\end{document}
